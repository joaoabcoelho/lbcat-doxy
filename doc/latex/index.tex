\subsection*{Introduction\+: }

Cat is the software for the control, configuration and test of the calorimeter electronics.

\hyperlink{tutorial}{C\+AT Tutorial} for a brief tutorial

\subsection*{Installation\+: }

You need to have cmake installed on your machine (this is on the standard repositories and probably already installed on your machine). C\+MT is not used anymore.

Just checkout a new working copy of the software, but requiring a specific branch which is called cmake.

Go to the main directory and adapt the setup.\+sh file. You should have only to modify the C\+A\+T\+P\+A\+TH variable, so that it points to the directory where the software is installed.

Type

\begin{quote}
source setup.\+sh \end{quote}


\begin{quote}
cmake . \end{quote}


\begin{quote}
make all \end{quote}


\begin{quote}
make install \end{quote}
The doxygen documentation can be built with the command

\begin{quote}
make doc \end{quote}


\subsection*{Starting Cat\+: }

To start cat you can run either in any place (alias defined in the setup.\+sh file)\+:

\begin{quote}
catcmd \end{quote}
for the command version

\begin{quote}
catgui \end{quote}
for the gui version. It is advised to create another directory where you start the software, so that all the user files/directory will stand in another place than the source and compilation files. An alias setup\+Cat is useful to launch the command

\begin{quote}
source setup.\+sh \end{quote}
with the proper path to the source.\+sh file. 